\makeatletter
\renewcommand\section{\@startsection{section}{2}{\z@}%
	{1.5ex \@plus1ex \@minus.2ex}%
	{.1ex \@plus.1ex}%
	{\normalfont\normalsize\bfseries}}								%% 以上4行,定义让section后自动断行
\makeatother

	\renewcommand{\chaptermark}[1]{\markboth{第 \thechapter\ 章\quad #1}{}}
	\CTEXsetup[name={第~,~章}, nameformat={\bfseries \fangsong \zihao{4}}, titleformat={\bfseries \fangsong \zihao{4}}, number={\arabic{chapter}}]{chapter}%一级标题四号仿宋
	\CTEXsetup[beforeskip={0pt}, afterskip={20pt}]{chapter}	%章的前后行距
	
	\CTEXsetup[format={\heiti \zihao{-4}}]{section}	%二级标题四号黑体
%	\CTEXsetup[beforeskip={2ex plus .5ex minus .2sex}, afterskip={0ex plus .2ex}]{section}
	
%	\CTEXsetup[format={\fangsong \zihao{-4}}]{subsection}%三级标题小四号黑体
	%\captionwidth{0.8\textwidth}
	%\changecaptionwidth
	\CTEXoptions[contentsname={\normalfont \heiti \zihao{4}目\quad 录}, figurename={图}, tablename={表}, bibname={\zihao{-4} \normalfont \heiti 参考文献}]
	
\usepackage[titles]{tocloft}
	\renewcommand{\cftdot}{$\cdot$}
	\renewcommand{\cftdotsep}{1.5}
	\setlength{\cftbeforechapskip}{10pt}
	\renewcommand{\cftchapleader}{\cftdotfill{\cftchapdotsep}}
	\renewcommand{\cftchapdotsep}{\cftdotsep}
	\makeatletter
	\renewcommand{\numberline}[1]{%
		\settowidth\@tempdimb{#1\hspace{0.5em}}%
		\ifdim\@tempdima<\@tempdimb%
		\@tempdima=\@tempdimb%
		\fi%
		\hb@xt@\@tempdima{\@cftbsnum #1\@cftasnum\hfil}\@cftasnumb}		%% 防止目录里章节号和章节名重叠
	\makeatother
	
\usepackage{amsmath}	%[fleqn]可以公式不居中
\usepackage{amsfonts}
\usepackage{amssymb}
\usepackage{graphicx}
\usepackage[paper=a4paper, top=25 mm, bottom=20mm, left=25 mm, right=30mm, head=5 mm, headsep=2.5 mm, foot=5 mm]{geometry}
\usepackage{tabularx}
\usepackage{txfonts}
\usepackage{booktabs} % 三线表
\usepackage{multirow} % 多列
\usepackage{fancyhdr}    % 页眉页脚
\usepackage[numbers,sort&compress]{natbib}
	\setlength{\bibsep}{0ex}  % vertical spacing between references
\usepackage{algorithm}
\usepackage{algorithmic}
\usepackage{float}
	%1方式,线型
	\floatstyle{ruled}
	%2环境、浮动方式、包含文件(类似toc,lof,lot)
	\newfloat{algorithm}{htbp}{loa}[chapter]
	%3 目录名称,类似-->\renewcommand*{\lstlistlistingname}{程~序}
	\floatname{algorithm}{算~法} %环境名,使用方法\begin{algorithm}caption{}.......\end{algorithm}

\usepackage{caption}  % 标题
	\renewcommand\captionfont{\zihao{-5} \heiti}
	\captionsetup{labelsep=space}
	\captionsetup{justification=centering}
%	\captionsetup[figure]{options}
%	\captionsetup{labelsep=colon}
	\setlength{\abovecaptionskip}{6pt}
	\setlength{\belowcaptionskip}{-10pt}
	
% Windows	\newCJKfontfamily\fzyt{FZYaoTi}
% Mac \newCJKfontfamily\fzyt{FZYTK--GBK1-0}
	\newCJKfontfamily\fzyt{FZYTK--GBK1-0}

%后面有了	\setmainfont{Times New Roman}

\usepackage{enumitem}
	\setenumerate[1]{itemsep=0pt,partopsep=0pt,parsep=\parskip,topsep=0pt}
	\setitemize[1]{itemsep=0pt,partopsep=0pt,parsep=\parskip,topsep=0pt}
	\setdescription{itemsep=0pt,partopsep=0pt,parsep=\parskip,topsep=0pt}

		
	\pagestyle{plain}

	
	\newcommand{\dif}{\mathop{}\!\mathrm{d}} 

\usepackage{microtype}	%	microtype 宏包可以改善了单词、字母的间距。它可能做了很多,但是大部分人察觉不到使用它之后文档的变化。但至少,加载了 microtype 之后,文档看起来更好,也更容易阅读。注意:如果有使用到字体宏包,需要将 microtype 宏包放在它们的后面,因为这个宏包对单词、字母的调整和字体是有关的。

\usepackage{siunitx}	%	siunitx 宏包大大简化了写作科技文的 TeX 命令,科技文写作中很大一部分是单位、数字。这个宏包添加了一些命令,比如 \num命令可以输出我们想要的各种方式的数字形式(比如科学记数法),而 \si 命令用来输出单位。我经常用到的命令是 \SI 和 \SIrange。比如 \SI{10}{\hertz} 输出为 “10Hz”(这能有效避免输入错误,我可能会写成 HZ 或者 hz 而不是 Hz)。\SIrange 命令多一个参数:\SIrange{10}{100}{\hertz} 输出为 “10Hz to 100Hz”。
